\documentclass[hidelinks]{article}
\usepackage{amsmath,amssymb,amsfonts,amsthm,amsbsy,amstext}
\usepackage[dvips]{graphicx}
\newcommand{\half}{\textstyle\frac{1}{2}}

\oddsidemargin 3mm
\evensidemargin 3mm
\topmargin -15mm
\textheight 670pt
\textwidth 450pt

\usepackage{hyperref}
\usepackage{url}
\usepackage{mathrsfs}

\newcommand{\R}{\mathbb R}
\newcommand{\Z}{\mathbb Z}
\newcommand{\Q}{\mathbb Q}
\newcommand{\N}{\mathbb N}
\newcommand{\C}{\mathbb{C}}

\begin{document}

\pagestyle{empty}


\noindent
{\bf\large Math 321   Assignment~6}
\hfill February 28, 2025
\hfill {Dr.\ G.\ Slade}

\medskip \noindent
This assignment is {\bf due in Canvas by 09:00 a.m.\ on Friday, March 7}.
\\
\emph{Late assignments are not accepted.}
Up to five points are given for overall quality of presentation.



\begin{enumerate}


\item
Let $X$ be a metric space with metric $d$.  Let $\alpha >0$.
A function $f \in {\cal C}(X)$ is called
\emph{H\"older continuous of exponent $\alpha$} if the quantity
\[
    N_\alpha(f) = \sup_{x\neq y}\frac{|f(x)-f(y)|}{d(x,y)^\alpha}
\]
is finite.
\begin{enumerate}
\item
Prove that if $X$ is compact then
$\{f \in {\cal C}(X) : \|f\| \le 1 \; \text{and}\; N_\alpha(f) \le 1\}$ is a compact subset
of ${\cal C}(X)$.
\\

\textit{Proof:}

Prove $\mathscr{F} = \{f \in {\cal C}(X) : \|f\| \le 1 \; \text{and}\; N_\alpha(f) \le 1\}$ is a compact subset of ${\cal C}(X)$.

Since each $f\in \mathscr{F}$ satisfies $||f||=\sup_{x\in X} |f(x)|\leq1$, thus $\mathscr{F}$ is definitely pointwise-bounded.

On the other hand, since $N_\alpha(f) = \sup_{x\neq y}\frac{|f(x)-f(y)|}{d(x,y)^\alpha}$, we can know that for any $x\neq y$, $|f(x)-f(y)|\leq N_\alpha(f)d(x,y)^\alpha$.

Since $N_\alpha(f)\leq1$, then for any $\epsilon>0$, there exists $\delta=\epsilon^{\frac{1}{\alpha}}$ such that if $d(x,y)<\delta$ then $|f(x)-f(y)|\leq d(x,y)^\alpha<\epsilon$. Thus, we have proved $\mathscr{F}$ is equicontinuous.

$\mathscr{F}$ met all the conditions of Arzelá-Ascoli Theorem, so any $\{f_n\}$ has a uniformly convergent subsequence in $\mathscr{F}$.

Next, we claim to prove $\mathscr{F}$ is closed then we can say $\mathscr{F}$ is compact.
 
Since \( f_n \) converges to \( f \) uniformly, we have
\[
\|f_n - f\| = \sup_{x \in X} |f_n(x) - f(x)|,  \lim_{n \to \infty}||f_n-f||=0.
\]
Because each \( f_n \) satisfies \( \|f_n\| \leq 1 \), we obtain
\[
\|f\| = \sup_{x \in X} |f(x)| = \lim_{n \to \infty} \sup_{x \in X} |f_n(x)| \leq 1.
\]
Thus, \( f \) still satisfies \( \|f\| \leq 1 \).

By the definition of Hölder continuity, each \( f_n \) satisfies
\[
|f_n(x) - f_n(y)| \leq d(x,y)^\alpha.
\]
Taking the limit, we obtain
\[
|f(x) - f(y)| = \lim_{n \to \infty} |f_n(x) - f_n(y)| \leq d(x,y)^\alpha.
\]
Thus, 
\[
N_\alpha(f) = \sup_{x \neq y} \frac{|f(x) - f(y)|}{d(x,y)^\alpha} \leq 1.
\]
We have proved that $\mathscr{F}$ is compact subset of $\mathcal{C}(X)$.
\item
Prove that $\{f \in {\cal C}([0,1]) : \|f\| \le 1 \}$ is \emph{not} a compact subset of
${\cal C}([0,1])$.
\\

\textit{Proof:}
Consider a counterexample: $f_n(x)=\cos(nx)$, then it is continuous, and satisfy $|f_n(x)|\leq1$. But it not equicontinuous, for example $x=0$ and $y=\frac{\pi}{2n}$, then $|\cos(nx)-\cos(ny)|=|\cos(0)-\cos(\frac{\pi}{2})|=1$.
\end{enumerate}

\item
Suppose that $f: \R\to \R$ can be uniformly approximated by polynomials.
Prove that $f$ must be a polynomial.
\\

\textit{Proof:}

Given \( f(x) \) can be uniformly approximated by a sequence of polynomials, suppose  there exists a sequence of polynomials $\{f_n(x)\}$:

\[
f_n(x) = a_k^{(n)} x^k + a_{k-1}^{(n)} x^{k-1} + \dots + a_1^{(n)} x + a_0^{(n)},
\]

and we know that $\| f_n - f \|_{\infty} \to 0,  \text{as } n \to \infty$.

Then for any $\epsilon>0$, there exists some $N \in \mathbb{N}$ such that any $m,n\geq N$ satisfy 
\[
\| f_n - f_m \|_{\infty} = \sup_{x \in \mathbb{R}} | f_n(x) - f_m(x) |<\epsilon
\]
by Cauchy criterion for uniform convergence.

$|f_n(x)-f_m(x)|=|(a_k^{(n)}-a_k^{(m)}) x^k + (a_{k-1}^{(n)}-a_k^{(m)}) x^{k-1} + \dots + (a_1^{(n)}-a_1^{(m)}) x + a_0^{(n)}-a_0^{(m)}|<\epsilon$.

Notice here, since the inequality valid for any $x\in \mathbb{R}$, we can see that for any $j=0,1,2...$, all terms with $x^{j}$ must be cancel out in order to satisfy the inequality, which means $|a_j^{(n)}-a_k^{(m)}|<\epsilon$. Each $\{a_j^{(n)}\}$ is Cauchy sequence.

Since $f_n(x) $ converges to $ f(x)$ uniformly, and each $f_n(x)$ is a polynomial of the form $\sum_{k=0}^\infty a_k^{(n)} x^j$, it follows that for every $x \in \mathbb{R}$, we have
\[
f(x) = \lim_{n \to \infty} f_n(x) = \sum_{j=0}^{\infty} a_j x^j.
\]
But we still need to show that this series terminates at a finite degree. That is, only finitely many $a_j$ are nonzero.

Since $f_n(x)$ is a sequence of polynomials, we note that eventually, all $f_n(x)$ have the same highest degree. Let $K$ be the highest power that appears in the approximating sequence for large enough $n$. That is, for all $n > N$, we have
\[
f_n(x) = a_K^{(n)} x^K + a_{K-1}^{(n)} x^{K-1} + \dots + a_1^{(n)} x + a_0^{(n)}.
\]
According to previous argument, we must have
\[
a_{K+1} = a_{K+2} = \dots = 0.
\]
Thus, the power series for $f(x)$ actually terminates at $j = K$, giving us
\[
f(x) = a_0 + a_1 x + a_2 x^2 + \dots + a_K x^K.
\]
Hence, we have proved $f$ must be a polynomial.





\item
Let $\alpha$ and $\beta$ be monotone non-decreasing continuous real-valued functions on $[0,1]$,
with $\alpha(0)=\beta(0) =0$.
Suppose that, for all $n =0,1,2,3,\ldots$,
\[
    \int_0^1 e^{-nx} d\alpha(x)
    =
    \int_0^1 e^{-nx} d\beta(x).
\]
\begin{enumerate}
\item
Prove that if $f:[0,1]\to \mathbb{R}$ is continuous then $\int_0^1 f(x) d\alpha(x)
= \int_0^1 f(x) d\beta(x)$.
\\

\textit{Proof:} 

Suppose $f$ is continuous function on $[0,1]$, by Weierstrass’ Theorem, there exists a sequence of polynomials $\{p_j(x)\}$ such that $P_j$ converges uniformly to $f$ on $[0,1]$.

Given a continuous function $g(e^{-x})=f(x)$, which is defined on $[e^{-1},1]$. We can also use a sequence of polynomials $\{g_j(t)\}$ converges uniformly to $g(t)$ on $[e^{-1},1]$ by Weierstrass's theorem.

For $x\in [0,1]$, we have $\sup_{x\in[0,1]}|g_j(e^{-x})-g_j(e^{-x})|=\sup_{t\in[e^{-1},1]}|g_j(t)-g_j(t)|$. Thus, $g_j(e^{-x})$ converges uniformly to $g(e^{-x})$.

According to for all $n=0,1,2,3...$
\[
    \int_0^1 e^{-nx} d\alpha(x)
    =
    \int_0^1 e^{-nx} d\beta(x).
\]
Since we know $g_j(e^{-x})=a_k^{(j)}e^{-kx}+a_{k-1}^{(j)}e^{-(k-1)x}+...+a_0^{(j)}$ is linear combination polynomial.

Thus for all $j=1,2,3...$
\[
    \int_0^1 g_j(e^{-x}) d\alpha(x)
    =
    \int_0^1 g_j(e^{-x}) d\beta(x).
\]
Then we know the fact by theorem 7.16:
\[
\lim_{j\to\infty}\int_0^1 g_j(e^{-x}) d\alpha(x)
    =\int_0^1 g(e^{-x}) d\alpha(x)=
    \lim_{j\to\infty}\int_0^1 g_j(e^{-x}) d\beta(x)=\int_0^1 g(e^{-x}) d\beta(x).
\]
Replace $g(e^{-x})$ with $f(x)$, we have proved that $\int_0^1 f(x) d\alpha(x)
= \int_0^1 f(x) d\beta(x)$.
\item
Does it follow that $\alpha(x)=\beta(x)$ for all $x \in [0,1]$?  Prove it or give a
counterexample.
\\

\textit{Proof:}

When $n=0$, $\int_0^1 e^{-0x} d\alpha(x)=\alpha(1)-\alpha(0)=\int_0^1 e^{-0x} d\beta(x)=\beta(1)-\beta(0).$

Since $\alpha(0)=\beta(0)=0$, thus $\alpha(1)=\beta(1)$.

For every fixed $x\in [0,1]$, Suppose
\[
f_x(t)  =
\begin{cases}
1, & 0 \leq t \leq x, \\
0, & t > x.
\end{cases}
\]
Then, its integral satisfies
\[
\int_0^1 f_x(t) d\alpha(t) = \alpha(x), \quad \int_0^1 f_x(t) d\beta(t) = \beta(x).
\]
However, since \( f_x(t) \) is not continuous, we cannot directly apply the results of part(a). 

Next, we define a sequence of continuous functions \( f_n(t) \) approximating \( f_x(t) \):
\[
f_n(t) =
\begin{cases}
1, & 0 \leq t \leq x, \\
n(x+\frac{1}{n}-t), & x  < t < x+\frac{1}{n} \\
0, & t \geq x+\frac{1}{n}.
\end{cases}
\]
This function has the following properties:
\begin{enumerate}
    \item $\lim_{n\to\infty}f_n(t)=f_x(t)$
    \item $f_n(t)$ is continuous on $[0,1]$, satisfies the results of part(a) for all $n=1,2,3...$
\end{enumerate}

Hence, we apply the results of part(a) for all $n=1,2,3...$:
\[
\int_0^1 f_n(t) d\alpha(t) = \int_0^1 f_n(t) d\beta(t).
\]
So when n goes to infinite, we obtain:
\[
\int_0^1 f_x(t) d\alpha(t) = \int_0^1 f_x(t) d\beta(t).
\]
Thus, we conclude
\[
\alpha(x) = \beta(x) \quad \forall x \in [0,1]
\]

\end{enumerate}

\item
Let ${\cal A}$ be an algebra of continuous real-valued functions on a compact metric space $K$.
Suppose that ${\cal A}$ separates the points of $K$.  Prove that the closure
$\overline{\cal A}$ of ${\cal A}$
consists of either: (i) all continuous functions $f:K\to \mathbb{R}$, or
(ii) all continuous functions $f$ on $K$ such that $f(p)=0$
for some fixed $p \in K$.
\\
Hint: in the second case it is useful to consider ${\cal A} + \mathbb{R}
= \{f+c : f \in {\cal A},\, c \in \mathbb{R}\}$.
\\

\textit{Proof:} 



Since \(\mathcal{A}\) is an algebra and separates points, we consider two cases:

Case1: If $\mathcal{A}$ vanishes at no points of $K$, then by the Stone-Weierstrass theorem, we have  
\[
\overline{\mathcal{A}} = \mathcal{C}(K).
\]  

Case2: If $\mathcal{A}$ vanishes at some points of $K$, suppose these fixed point $p \in K$ such that every function $f \in \mathcal{A}$ satisfies $f(p) = 0$. That is, $\mathcal{A}$ consists of functions that vanish at $p$. We claim to show that $\overline{\mathcal{A}}$ consists of all continuous functions $f\in K$ such that $f(p)=0$.

Consider the set ${\cal A} + \mathbb{R}
= \{f+c : f \in {\cal A},\, c \in \mathbb{R}\}$. 

This set contains all functions in $\mathcal{A}$ shifted by a constant. If $\mathcal{A}$ separates points, then for any distinct $x, y \in K$, there exists a function $g \in \mathcal{A}$ such that $g(x) \neq g(y)$. Given that $\mathcal{A}$ vanishes at $p$, the functions in $\mathcal{A} + \mathbb{R}$ separate points and also include a nonzero constant function.

Thus, $\mathcal{A} + \mathbb{R}$ satisfies the conditions of the Stone-Weierstrass theorem, $\overline{\mathcal{A}+\mathbb{R}}=\mathcal{C}(K)$.

Since $\mathcal{A}$ consists of functions that vanish at $p$, its closure $\overline{\mathcal{A}}$ must consist precisely of those functions in $\mathcal{C}(K)$ that still vanish at $p$. Thus, $\overline{\mathcal{A}}=\{f\in \mathcal{C}(K)| f(p)=0\}$.

Hence, the closure
$\overline{\cal A}$ of ${\cal A}$
consists of either: (i) all continuous functions $f:K\to \mathbb{R}$, or
(ii) all continuous functions $f$ on $K$ such that $f(p)=0$
for some fixed $p \in K$.



\item
Let $K = \{ z \in \mathbb{C} : |z|=1\}$ and let
\[
    {\cal A} = \left\{
    f : K \to \mathbb{C} \mid f(e^{i\theta}) = \sum_{n=0}^N c_n  e^{in\theta}
    \text{  for some $N=0,1,2,\ldots$ and some $c_0,\ldots,c_N \in \mathbb{C}$}
    \right\}.
\]
Convince yourself (do not hand in this part) that ${\cal A}$ is an algebra that
separates points and vanishes at no point in $K$.
\\
Prove that there are continuous functions on $K$ which are not in the uniform closure
of ${\cal A}$.
\\
Hint: prove that $\int_0^{2\pi} f(e^{i\theta}) e^{i\theta} d\theta = 0$ for all $f$ in
the uniform closure of ${\cal A}$.
\\

\textit{Proof:} 

According to the definition of $f(x)$, $\int_0^{2\pi} f(e^{i\theta}) e^{i\theta} d\theta = \int_0^{2\pi} \sum_{n=0}^N c_n e^{in\theta} e^{i\theta} d\theta =\int_0^{2\pi} \sum_{n=0}^N c_n  e^{i(n+1)\theta} d\theta $. 

Since $e^x=\cos x+i\sin x$, then $\int_0^{2\pi} f(e^{i\theta}) e^{i\theta} d\theta = \int_0^{2\pi} \Bigg[(\cos \theta +i\sin \theta)+(\cos 2\theta +i\sin 2\theta)+...+\cos N\theta +i\sin N\theta\bigg] d\theta$. 

Since this is a finite linear integral sum, and $\int_0^{2\pi} \cos xdx=\int_0^{2\pi} \sin xdx=0$.

Thus $\int_0^{2\pi} f(e^{i\theta}) e^{i\theta} d\theta = 0$ for all $f$ in
the uniform closure of ${\cal A}$.

Next, we consider a continuous function $f_1(z)=\frac{1}{z}$ on K, which is only meaningless for $z=0$.
Assume $f_1(z)$ in the uniform closure of ${\cal A}$, which means $f_1(e^{i\theta})=e^{-i\theta}$ and $\int_0^{2\pi} f_1(e^{i\theta})e^{i\theta}d\theta=\int_0^{2\pi} 1d\theta=2\pi\neq0$, Contradicts the previous statement.

Hence, there are continuous functions on $K$ which are not in the uniform closure
of ${\cal A}$.


\end{enumerate}









\end{document}
